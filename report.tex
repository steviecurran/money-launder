\documentclass[11pt]{article} %latex2e
\usepackage{etex}
\reserveinserts{28} %http://tex.stackexchange.com/questions/38607/no-room-for-a-new-dimen

\usepackage{natbib,caption,hyperref,graphics,pdfcolmk,framed,empheq,mathtools,mdframed,wasysym}
\usepackage[font=small,format=plain,labelfont=bf,textfont=sf]{caption}
\usepackage[dvipsnames]{xcolor} % % https://en.wikibooks.org/wiki/LaTeX/Colors#The_68_standard_colors_known_to_dvips
\usepackage{soul} % highlighting text
\setstcolor{red} % For \st with soul
\usepackage{graphicx,wrapfig}
\usepackage[floatrow]{chemstyle}
\usepackage[thicklines]{cancel} 
  \renewcommand{\CancelColor}{\color{red}}
\usepackage[T1]{fontenc}
\usepackage{natbib,caption,hyperref,graphics,pdfcolmk,listings,textcomp}
% %\RequirePackage[hidelinks]{hyperref}
\hypersetup{
  colorlinks   = true, %Colours links instead of ugly boxes
  urlcolor     = blue, %Colour for external hyperlinks
  linkcolor    = black, %Colour of internal links
  citecolor   = black %Colour of citations
}

\setlength{\textwidth}{170mm}
\setlength{\textheight}{230mm} % WAS 245 BUT NO GOING FANCY
\setlength{\topmargin}{-20mm}
\setlength{\parindent}{0mm}
\setlength{\parskip}{3mm}
\setlength{\oddsidemargin}{0mm}

%%%%%%%%%%%%%%%%%%%%%%%%%%%%%%%%%%%%%%%%%%%%%%%%%%%%%%%%%%%
\newcommand*\whiteredbox[1]{%
\setlength{\fboxrule}{1.5pt}     % yes, this works!
\fcolorbox{red}{white}{\hspace{1em}#1\hspace{1em}}
}  %#1 is arguement
 
\newcommand*\whitebluebox[1]{%
\setlength{\fboxrule}{1.5pt}
\fcolorbox{blue}{white}{\hspace{0.5em}#1\hspace{0.5em}}
}   
 
\newcommand*\whitegreenbox[1]{%
\setlength{\fboxrule}{1.5pt}
\fcolorbox{green}{white}{\hspace{0.5em}#1\hspace{0.5em}}
}   
\setlength\fboxsep{0.5cm} %which will increase the space around the formula in all directions
\renewcommand{\familydefault}{\sfdefault}
%\usepackage[helvet]{sfmath} 
\renewcommand{\baselinestretch}{1.2}
%\pagestyle{headings} 

\def\lapp{\ifmmode\stackrel{<}{_{\sim}}\else$\stackrel{<}{_{\sim}}$\fi}
\def\gapp{\ifmmode\stackrel{>}{_{\sim}}\else$\stackrel{>}{_{\sim}}$\fi}

\setlength{\footskip}{25pt} 
\setlength{\fboxrule}{2pt}
\newcommand{\mathcolorbox}[2]{\colorbox{#1}{$\displaystyle #2$}}

% https://tex.stackexchange.com/questions/132783/how-to-write-checkmark-in-latex
\usepackage{amssymb} %TICKS

%%%%%%%%%%% GETTING FANCY %%%%%%%%%%%%%%%%%%%%%%%%%%%%%
%\linespread{1.5} %regulate line spacing
\usepackage{fancyhdr}
\setlength{\headheight}{55pt}
\renewcommand{\headrulewidth}{2pt}% 2pt header rule
\renewcommand{\headrule}{\hbox to\headwidth{%
    \color{Blue}\leaders\hrule height \headrulewidth\hfill}     
}

\usepackage{titlesec} % coloured section headings and better spacing
\titleformat{\section}{\normalfont\bfseries}{\thesection}{1em}{}
\titlespacing*{\section}{0pt}{*0}{0pt}
\titleformat{\section}
{\color{Plum}\normalfont\Large\bfseries}
{\color{Plum}\thesection}{1em}{}
\titleformat{\subsection}
%{\color{Plum}\normalfont\large}
{\color{BlueViolet}\normalfont\large\bfseries}
{\color{BlueViolet}\thesubsection}{1em}{}
\titleformat{\subsubsection}
{\color{BlueViolet}\normalfont\normalsize\bfseries}
{\color{BlueViolet}\thesubsubsection}{1em}{}
%%%%%%%%%%%%%%%%%%%%%%%%%%%%%%%%%%%%%%%%%%%%%

\begin{document}


%\vspace*{3cm}
\Large
\begin{center}
\setlength{\fboxrule}{1.5pt} 
\fcolorbox{blue}{white}{\textbf{\textit{Machine Learning Prediction of Money Laundering Customers}}}\\
\end{center}
%\vspace{1cm}
\large

\begin{center}
  {\bf S. J. Curran}
  %\date
\end{center}

\vspace{1cm}
\normalsize

%}}

%\vspace{3cm}
\normalsize

\begin{center}
\section*{Summary}
\end{center}

We present a machine learning model which our client, an online gambling platform, can use to detect customers (players)
using their infrastructure to launder money. Although we are limited to a sample
containing only 1752 suspect players, we
report a mean accuracy of 81\% in the prediction of a player engaging in suspect activity. The model requires
only seven of the 17 available features, potentially increasing its applicability
to more players, as well as possibly providing a much larger training sample which could increase the model accuracy.
%\newpage
\small

\tableofcontents

\newpage
\pagenumbering{arabic}
\normalsize

\pagestyle{fancy}
\lhead{{\color{Blue}\fancyplain{}\em Suspected Money Laundering Players}}
\fancyfoot{}
\fancyfoot[R]{\thepage} 
\pagenumbering{arabic} 
%\renewcommand\thepage{\romannumeral\numexpr\value{page}-1\relax} % NUMBER AFTER CONTENTS

\section{Introduction}

The client, an online gambling platform, requires a machine learning model with which to detect customers (players)
who use the client's infrastructure for laundering money through apparent gambling activities.
The findings are summarised above, with the following report containing the specific details
regarding the building of the model.

\section{Analysis}

\subsection{Pre-processing}

\subsubsection{The data}

The client has supplied five datasets:
\begin{enumerate}
\item {\tt details.csv} -- the 207\,398 players’ basic details. The file is known to have formatting issues.

\item {\tt payments.csv} -- the payment activities of the players over the last 30 days.

\item {\tt profiling.csv} -- additional information regarding the players' activity over the same period. 

\item {\tt suspect.csv} -- a list of 1752 players already flagged for suspect activity.

\end{enumerate}

The fields are (in order of appearance):
\begin{tabbing}
{\tt ID} ~~~~~~~~  ~~~~~~~~  ~~~~~~~~  ~~~~~~~~  ~~~~~~~~ \= the unique ID of the player\\
{\tt Life\_Time} \> how many years the player has been a client \\ 
{\tt Age} \> the client's age in years\\
{\tt Is\_Retail} \> whether the player is also a retailer\\
{\tt Is\_CRM\_Email} \> whether the player has been contacted by us \\
%\>  ~~~ by the Customer Relationship Management team\\
{\tt CountDeposit} \> number of deposits\\
{\tt CountWithdrawal} \> number of withdrawals\\
{\tt TotalDeposits} \> value of deposits\\
{\tt TotalWithdrawal} \> value of withdrawals\\
{\tt CountPaymentMethod} \> number of different deposit/withdrawal methods \\
{\tt DifferentMethodWithdrawals} \> value of different deposit/withdrawal methods \\
{\tt Multi\_Device} \> 0 flags $<3$ devices and  1 flags $\geq3$ devices used for payment\\
{\tt IP\_Counts} \> number of different IP addresses used for transactions\\
{\tt e\_11234, e\_23456, e\_34454}\> aggregated variables describing the players' activity\\
~~~~~~~~~~~~~~~~~~{\tt e\_43568, e\_64645} \> \\
\end{tabbing}

\subsubsection{First look at the data}

Using {\tt pre-process.py}, we see that {\tt details.csv} does indeed have formatting issues:

\small
{\color{blue}
\begin{verbatim}
                                ID|Life_Time|Age|Is_Retail|Is_CRM_Email
NaN db=123 | 25 | 100001096=id     | | 1 )| | | 33 | | )| 1 | | | 0 | 
    db=123 | 25 | 100002057=id     | | 1 )| | | 44 | | )| 0 | | | 0 | 
 ...                                                                ...
    db=123 | 25 | 899994603=id     | | 3 )| | | 35 | | )| 0 | | | 0 | 
    db=123 | 25 | 899998254=id     | | 0 )| | | 39 | | )| 0 | | | 0 | 
[207398 rows x 1 columns]
\end{verbatim}
}
\normalsize
This was easily fixed in the {\tt emacs} editor\footnote{Command line methods such as {\tt sed}, {\tt awk}, etc.
  could also have been used.} and saved as {\tt details\_fixed.csv}

\small
{\color{blue}
\begin{verbatim}
               ID  Life_Time  Age  Is_Retail  Is_CRM_Email
0       100001096          1   33          1             0
1       100002057          1   44          0             0
...           ...        ...  ...        ...           ...
207396  899994603          3   35          0             0
207397  899998254          0   39          0             0

[207398 rows x 5 columns]
                 ID      Life_Time            Age      Is_Retail   Is_CRM_Email
count  2.073980e+05  207398.000000  207398.000000  207398.000000  207398.000000
mean   5.008909e+08       2.603342      41.490410       0.410023       0.418042
std    2.306288e+08       3.122641      13.123745       0.491839       0.493238
min    1.000011e+08       0.000000      18.000000       0.000000       0.000000
max    8.999983e+08      21.000000     120.000000       1.000000       1.000000
\end{verbatim}
}
\normalsize

Looking at {\tt payments.csv},
\small
{\color{blue}
\begin{verbatim}
               ID  CountDeposit  ...  CountPaymentMethod  DifferentMethodWithdrawals
0       100001096            86  ...                   1                      2359.8
1       100002057            26  ...                   1                         0.0
...           ...           ...  ...                 ...                         ...

207396  899994603             1  ...                   1                         0.0
207397  899998254             5  ...                   1                         0.0
\end{verbatim}
}
\normalsize
this has the customer ID ({\tt ID}) in common with {\tt details.csv}, as does {\tt profiling.csv}.
\newpage
\small
{\color{blue}
\begin{verbatim} 
               ID  Multi_Device  IP_Counts   e_11234   e_23456   e_34454   e_43568  e_64645
0       100001096             1          3   5599.89    986.36    283.05    662.98     0.40
1       100002057             1          1   9669.93  17181.08  20905.39  38635.13     3.22
...           ...           ...        ...       ...       ...       ...       ...      ...
207396  899994603             1          2   7522.49  16485.09  37475.16  37475.16     3.15
207397  899998254             1          1   9229.69   3589.55    578.35  31473.88     1.22
\end{verbatim}
}
\normalsize

We therefore merged these datasets by {\tt ID} and checking the merged data we find no missing values.
\small
{\color{blue}
\begin{verbatim} 
             Life_Time            Age  ...        e_34454        e_43568        e_64645
count    207398.000000  207398.000000  ...  207398.000000  207398.000000  207398.000000
mean          2.603342      41.490410  ...   14666.698999   18722.672481       1.516427
std           3.122641      13.123745  ...   13152.810961   15426.545697       1.101400
min           0.000000      18.000000  ...       0.000000       0.000000       0.000000
25%           0.000000      31.000000  ...    3454.317500    3320.920000       0.490000
50%           1.000000      39.000000  ...   10207.420000   14811.630000       1.350000
75%           4.000000      51.000000  ...   25097.677500   34063.392500       2.670000
max          21.000000     120.000000  ...   70086.760000   56891.910000       4.310000
[8 rows x 18 columns]
\end{verbatim}
}
\normalsize


Finally, looking at {\tt suspect.csv}, this only contains only two fields,
the player ID and a positive flag ({\tt Target\_ml} = 1) for suspected laundering activity.
\small
{\color{blue}
\begin{verbatim} 
                 ID  Target_ml
count  1.752000e+03     1752.0
mean   4.995049e+08        1.0
std    2.322098e+08        0.0
min    1.000955e+08        1.0
max    8.997882e+08        1.0
\end{verbatim}
}
\normalsize
All 1752 of these players were also in the merged file, which was absent a suspect activity flag.
These flags were added to
the merged file and, assuming that the $207\,389-1752 = 205\,637$
remaining players are non-suspect, we flagged these as {\tt Target\_ml} = 0.
\small
{\color{blue}
\begin{verbatim}
               ID  Life_Time  Age     ...   e_23456   e_34454   e_43568  e_64645  Target_ml
0       100001096          1   33     ...    986.36    283.05    662.98     0.40          0
1       100002057          1   44     ...  17181.08  20905.39  38635.13     3.22          0
...           ...        ...  ...     ...       ...       ...       ...      ...        ...
207396  899994603          3   35     ...  16485.09  37475.16  37475.16     3.15          0
207397  899998254          0   39     ...   3589.55    578.35  31473.88     1.22          0

[207398 rows x 19 columns]

             Life_Time            Age  ...        e_43568        e_64645      Target_ml
count    207398.000000  207398.000000  ...  207398.000000  207398.000000  207398.000000
mean          2.603342      41.490410  ...   18722.672481       1.516427       0.008448
std           3.122641      13.123745  ...   15426.545697       1.101400       0.091522
min           0.000000      18.000000  ...       0.000000       0.000000       0.000000
max          21.000000     120.000000  ...   56891.910000       4.310000       1.000000
\end{verbatim}
}
\normalsize
We then save these data to  {\tt all\_data.csv} to be used in the machine learning.

\subsubsection{Distributions}

In order to eyeball any obvious differences between the suspect and non-suspect players, we
plot the distributions of the non-binary features.


  \includegraphics[angle=0,scale=0.52]{media/histo_Life_Time.eps}
  \includegraphics[angle=0,scale=0.52]{media/histo_Age.eps}

  \includegraphics[angle=0,scale=0.52]{media/histo_CountDeposit.eps}
  \includegraphics[angle=0,scale=0.52]{media/histo_CountWithdrawal.eps}

  From these we see that the suspect players features (red histograms) mostly
  overlap with the non-suspect plays (black histograms), demonstrating that
  these would be difficult to distinguish from one another based the distributions alone.
  Note also that some of the non-suspect players are suspiciously long-lived.

  
  \includegraphics[angle=0,scale=0.52]{media/histo_TotalDeposits.eps}
  \includegraphics[angle=0,scale=0.52]{media/histo_TotalWithdrawal.eps}

  \includegraphics[angle=0,scale=0.52]{media/histo_CountPaymentMethod.eps}
  \includegraphics[angle=0,scale=0.52]{media/histo_DifferentMethodWithdrawals.eps}

  \includegraphics[angle=0,scale=0.52]{media/histo_IP_Counts.eps}
 \includegraphics[angle=0,scale=0.52]{media/histo_e_11234.eps}

 \includegraphics[angle=0,scale=0.52]{media/histo_e_23456.eps}
 \includegraphics[angle=0,scale=0.52]{media/histo_e_34454.eps}

 \includegraphics[angle=0,scale=0.52]{media/histo_e_43568.eps}
\includegraphics[angle=0,scale=0.52]{media/histo_e_64645.eps}

\subsection{Machine Learning}

\subsubsection{Initial models and tests}
\label{moii}

In {\tt ML.py} we initially test four types of common, but different, ML classifiers:
\begin{enumerate}
\item {\em Logistic Regression} (LR): This is analogous to multi-variable linear regression,
  but instead of a fit yields a binary result. Thus, it is particularly suited to this problem
  where we are classifying a player as suspect or not. The algorithm compresses a
  linear combination of several variables (features) with a logistics sigmoid to
 yield a value of between 0 and 1. For the binary model, the prediction is labelled with one of these two end values,
 depending upon its probability ($0\rightarrow1$), or odds ($0\rightarrow\infty$)

\item {\em k-Nearest Neighbour} (kNN): This  algorithm maps the variables to a feature space and then compares
the Euclidean distance between a test point and its $k$ nearest neighbours. It then assigns a
weighted combination of the target values with the nearest neighbours in order to place the test
object in a group. The kNN algorithm is relatively computationally expensive and, like logistic
regression, is sensitive to outliers. A further disadvantage is that irrelevant features can lead the
learning astray.

\item {\em Support Vector Classifier} (SVC): This constructs a hyperplane in a high dimensional space in order to perform classification,
regression and outlier detection. 
Support vectors are points that reside closest to the hyperplane and 
in binary classification the training maximises the distance between the two categories. Further data are
transformed into the same space and assigned a category based upon where in the space the are located.
Although the support vector machine classifier is computationally fast, it is not suitable 
for noisy data (overlapping features) nor large data sets.

\item {\em Decision Tree Classifier} (DTC): Like the other algorithms, decision trees can be used for both classification and regression.  The algorithm builds a
classification model based upon a tree structure, which branches the data-set (top node) into smaller subsets (child
nodes), according to a predefined decision boundary.  With one node on either side of the boundary, the process is
iterated through further branching until a predefined stopping criterion in reached. DTC has the advantage that it is
not as sensitive to outliers as some of the other algorithms, although the tree choices can be biased due to the
sequential nature of the algorithm.  Over-fitting can also be a problem, which can be mitigated by limiting the maximum
tree depth.
\end{enumerate}

As stated above, there are only 1752 suspect players and so from raw statistics alone we can ascertain that
there is a probability of $100\times1752/207\,389 = 0.84$\% of the player being suspect.
In order to prevent the ML blindly yielding an apparent $100 - 0.84 \approx 99$\% accuracy from probability alone, 
we randomly selected 1752 of the remaining 205\,637 non-suspect players for the binary testing.

Training on 80\% of the $2\times1752$ strong sample (2803 players) and testing on the other 20\% (701), using the default
values of the algorithms, we obtain the following cross-validation scores (in percent).
\small
{\color{blue}
\begin{verbatim}
For a 0.2 test fraction (2803 train & 701 test) LR score = 76.597
For a 0.2 test fraction (2803 train & 701 test) KNN score = 78.880
For a 0.2 test fraction (2803 train & 701 test) SVC score = 81.662
For a 0.2 test fraction (2803 train & 701 test) DTR score = 80.448
\end{verbatim}
\normalsize
}

\subsubsection{Model optimisation}
\label{mo}

We then optimise each of the models via:
\begin{itemize}
\item {\em Logistic regression} -- using {\tt sklearn GridSearchCV} giving {\tt LR.py}\footnote{Warning, this takes a while!}, 
  {\tt C=0.004833}, {\tt solver='newton-cg'}.
  
\item {\em Support Vector Classifier} -- using {\tt sklearn GridSearchCV} giving {\tt SVC.py}, {\tt C=1, gamma=0.1}.

\item {\em k-Nearest Neighbour} \& {\em Decision Tree Classifier} -- we iterated the number of nearest neighbours and
  maximum depths, giving {\tt n\_neighbors} = 17 and {\tt max\_depth} = 6, respectively.
  \end{itemize}

\includegraphics[angle=0,scale=0.55]{media/kNN-results.eps}
\includegraphics[angle=0,scale=0.55]{media/DTC-results.eps}

Re-running {\tt ML.py} with these parameters we obtain
\small
{\color{blue}
\begin{verbatim}
For a 0.2 test fraction (2803 train & 701 test) LR score = 76.026
For a 0.2 test fraction (2803 train & 701 test) KNN score = 79.987
For a 0.2 test fraction (2803 train & 701 test) SVC score = 81.485
For a 0.2 test fraction (2803 train & 701 test) DTR score = 80.450
\end{verbatim}
\normalsize
}
indicating that the ``default'' values were close to optimal and that SVC is the best model.
%However, due to the small sample size, these numbers can vary a lot in relative terms between runs.
%(although essentially unchanged, see Sect.~\ref{}). 

Below we show the confusion matrix for the run of the SVC algorithm.

\begin{tabular}{@{}l  cccc  @{}}
                  & \multicolumn{2}{c}{\bf Predicted training} & \multicolumn{2}{c}{\bf Predicted validation} \\
                  & {\bf Suspect} & {\bf Non-suspect} & {\bf Suspect} & {\bf Non-suspect}\\
{\bf Actual suspect} & 1372 & 410 & 321 & 109 \\
{\bf Actual non-suspect} & 42 & 979 & 17 & 254 \\
\end{tabular}

The validation data yield 321 true positives (suspect) and 254 true negatives (non-suspect),
compared to 17 false positives 
and 109 false negatives. This gives a test score of (321 + 254)/(321 + 254 + 17 + 109) = 0.82, the cross-validation score.

However, as above, these tests are based upon a single instance of the
models and, given the small sample, could be subject to the choice of the 1752 non-suspect players.
Using {\tt ML-loop.py} we run 1000 trials randomising the sample of non-suspect players from the pool of
205\,646 each time.\footnote{From 207\,398 total minus the 1752 suspect players.}

Summarising the results in the histograms below, we see
that the SVC algorithm performs best with a mean cross-validation
score of $81.23\pm0.46$\%.

\includegraphics[angle=0,scale=0.65]{media/ML_loops=1000.csv.eps}

\subsubsection{Feature importance}
\label{fi}

Below, from {\tt ML.py}, we show the feature importance for each algorithm in
descending order from a single run.
\small
{\color{blue}
\begin{verbatim}
 LogisiticRegression score = 76.524         Support Vector Classifier score = 81.55
                    Feature  Importance                         Feature  Importance
         CountPaymentMethod    0.053086                         e_23456    0.067499
                    e_64645    0.024331                         e_64645    0.055369
                    e_43568    0.022975                         e_43568    0.051445
                  IP_Counts    0.016625                       IP_Counts    0.017838
                    e_23456    0.010275                       Is_Retail    0.017339
                    e_34454    0.009989                         e_34454    0.016768
                  Is_Retail    0.006136              CountPaymentMethod    0.015912
 DifferentMethodWithdrawals    0.005066      DifferentMethodWithdrawals    0.012986
               Multi_Device    0.003568                         e_11234    0.012772
                  Life_Time    0.003211                       Life_Time    0.011559
               CountDeposit    0.001570                   TotalDeposits    0.010917
            CountWithdrawal    0.001284                    CountDeposit    0.008919
                    e_11234    0.000357                 TotalWithdrawal    0.008205
               Is_CRM_Email    0.000214                 CountWithdrawal    0.006778
                        Age    0.000143                    Multi_Device    0.006422
              TotalDeposits   -0.000571                    Is_CRM_Email    0.005351
            TotalWithdrawal   -0.001070                             Age    0.005351
                                            
 KNearest score = 78.843                      DecisionTreeClassifier score = 78.701  
                    Feature  Importance                         Feature  Importance
                    e_64645    0.043453                 CountWithdrawal    0.231038
                    e_23456    0.038673                   TotalDeposits    0.140564
                    e_43568    0.035391      DifferentMethodWithdrawals    0.070781
         CountPaymentMethod    0.026543                    CountDeposit    0.068783
                  Is_Retail    0.026258                         e_23456    0.048519
                    e_11234    0.016054                         e_64645    0.040314
                    e_34454    0.014413                       IP_Counts    0.022333
                  IP_Counts    0.012558                         e_43568    0.020407
               Is_CRM_Email    0.009704                 TotalWithdrawal    0.018480
                        Age    0.009633                         e_34454    0.012130
               Multi_Device    0.008348                       Life_Time    0.010560
                  Life_Time    0.008063                         e_11234    0.001998
               CountDeposit    0.005280                             Age    0.000999
            CountWithdrawal    0.003068                    Is_CRM_Email    0.000999
              TotalDeposits    0.002426                    Multi_Device    0.000000
            TotalWithdrawal    0.001927                       Is_Retail    0.000000
 DifferentMethodWithdrawals    0.001213              CountPaymentMethod    0.000000
\end{verbatim}
\normalsize
}
% rename 's/ML_1/ML/g' *
These were found to be roughly consistent, although there was some variation between different runs.
Focusing on the best performing (SVC) algorithm, we found that the score remained
above 80\% if we retained only the top seven features. Therefore, we test each classifier with only
the seven most important features listed for each one above.

The results are summarised in the following table, 
 \small
 \begin{table}
   {\setlength{\tabcolsep}{3pt} % NARROW COLUMN SPACING
  \begin{tabular}{l | c c c c c r}
    \hline
{\bf Feature}                      & \multicolumn{4}{c}{\bf Top Seven Classifier Features}  &      {\bf Number of} \\  
                                    & {\bf LR}   & {\bf SVC}   & {\bf kNN} & {\bf DTC}    & {\bf Occurrences}      \\
\hline
{\tt e\_11234}                     &            &             & \checkmark    &            & 1 \\
{\tt e\_23456}                     & \checkmark & \checkmark  & \checkmark    &            & 3\\            
{\tt e\_34454}                     & \checkmark & \checkmark  &  \checkmark   &            & 3\\ 
{\tt e\_43568}                     & \checkmark & \checkmark  & \checkmark    &            & 3\\
{\tt e\_64645}                     & \checkmark & \checkmark  & \checkmark    &  \checkmark& 4\\
{\tt CountDeposit}                 &            &             &               & \checkmark & 1\\
{\tt CountPaymentMethod}           &\checkmark  & \checkmark  & \checkmark    &            & 3 \\
{\tt CountWithdrawal}              &            &             &               & \checkmark & 1\\
{\tt DifferentMethodWithdrawals}   & \checkmark &             &               & \checkmark & 2\\
{\tt Is\_Retail}                   & \checkmark & \checkmark  & \checkmark    &            & 3\\    
{\tt IP\_Counts}                   &            & \checkmark  &                & \checkmark & 2\\
{\tt TotalDeposits}                &            &             &                 & \checkmark & 1 \\
\hline
{\bf Classifier mean score [\%]}                  &            &              &               &   & {\bf All features}    \\
\hline
LR                 &    $74.95\pm0.62$        &   $74.92\pm0.62$     &  $74.93\pm0.61$  & $72.01\pm0.70$  & $76.07\pm0.58$\\
SVC                   & $80.34\pm0.46$            &  $80.33\pm0.45$   & $80.47\pm0.45$  & $79.51\pm0.47$  & \hl{$81.23\pm0.46$}\\
kNN                  & \hl{$80.44\pm0.52$}            & \hl{$80.41\pm0.52$}  &  \hl{$81.04\pm0.47$} & $78.60\pm0.57$ & $78.92\pm0.55$ \\
DTC                  &   $78.68\pm0.68$          &  $78.71\pm0.65$   &  $78.55\pm0.65$   & \hl{$79.72\pm0.64$}  & $79.86\pm0.71$ \\
\hline
  \end{tabular}
  }
 \end{table}  
\normalsize where we see the top features are similar between the LR, SVC and kNN classifiers, with
only {\tt e\_64645} being common to all.  Bear in mind, however, this is based on the single run
shown above. Again, running 1000 trials, where we run {\em all} top seven classifier features for
each one ({\tt ML-loop\_feat.py}), we see that the results are largely insensitive to the choice of
the seven features, suggesting that the most important features are a subset of those above.

We also see that the LR classifier remains the worst performing for these data, in addition to  a
marked improvement in the kNN performance.  In fact, from the highlighted values, we see that this
is the best performer for the LR, SVC and kNN features.  As was noted in Sect.~\ref{moii}, the
effectiveness of the kNN algorithm is susceptible to irrelevant features which may be what we are seeing
here.

\subsection{Deep Learning}

Although, due to having only 1752 suspect players available for testing, the sample is likely too
small to yield reliable predictions from a deep learning algorithm, we nevertheless test this in
case any further insight can be gained.

In {\tt DL.py} we use {\sf TensorFlow}, with the testing of various hyper-parameters giving the best results with two
{\em ReLu} layers comprising $\approx50$ neurons each. 

\begin{minipage}[l]{0.40\linewidth}

 In order to get an accurate representation of the results, in {\tt DL-loop.py} we run the deep
 learning algorithm 100 times using all of features as well as the top seven for the kNN
 classifications (see above). From this we see that, although the means of the scores are similar to
 the machine learning models, the spreads are significantly wider, meaning that relying on this deep
 learning model carries a significant risk of being much less accurate than the machine learning.
  \end{minipage}\hfill
\begin{minipage}[l]{0.55\linewidth}
 \includegraphics[angle=0,scale=0.6]{media/DL_histos.eps}
  
\end{minipage}

%\newpage
\section{Discussion}

%\subsection{Results}
 We summarise the steps taken in the flow diagram below.

\begin{minipage}[l]{0.5\linewidth}
\includegraphics[angle=0,scale=0.4]{steps.eps}
\end{minipage}\hfill
\begin{minipage}[l]{0.45\linewidth}
 
  \begin{itemize}

  \item In {\tt pre-process.py} we combine the players details ({\tt details\_fixed.csv}) with their
    payment activities ({\tt payments.csv}) and other additional information ({\tt
      profiling.csv}). Using {\tt suspect.csv} we then flag those with known suspect activity.

  \item In {\tt ML.py} we select the 1752 suspect players in addition to a random sample of 1752
    non-suspect. Four different classification algorithms are tested on the sample of 3504 and then
    optimised.

  \item Selecting such a relatively small number of non-suspects from a pool of 205\,646 may lead to
    a single trial being unrepresentative. Therefore, in {\tt ML-loop.py} we randomise the selection
    of the 1752 non-suspect players and re-run the algorithm. After 1000 runs, we find the Support
    Vector Classifier to be the best performer with a mean score of $81.23\pm0.46$\%.
    \end{itemize}
\end{minipage}
\begin{itemize}
\item Finally, we look at the relative importance of each feature and find that retaining only the
  seven most important has little detrimental effect. In the case of the kNN classifier the
  performance reaches the same level as the full 17 feature SVC classifier.
   \end{itemize}

Note that the requirement of only seven features has the potential to vastly increase the sample
sizes and number of players which can be tested for likely money laundering.\footnote{The SVC and
kNN models, which use the top seven kNN features, have been saved in {\tt KNN-0.818.pickle} and {\tt
  SVC-0.812.pickle}, which as the names suggest, have training scores of 81.8\% and 81.2\%,
respectively.}

%% \subsection{Implemtation}

%% Although we have no unused data, 
%% we can perform a sanity check on the ML models by checking the predicted values against the actual 
%% {\tt Target\_ml} flags. This will also give a feel of how the client would utilise the models.

%% After saving the SVC and kNN models which use the top seven kNN features ({\tt top7.py}),
%% we return to the data, where we select a number of suspect and
%% non-suspect players at random from the 1752 of each available.
%% We then remove the target ({\tt Target\_ml} flag) and, upon scaling the
%% remaining features, predict whether the players are suspect or not ({\tt ML\_predictions.py}).

%% For this we use the SVC and kNN algorithms, with only the final seven features (Sect.~\ref{fi})\footnote{For the sake of space, we have renamed {\tt CountPaymentMethod} as  {\tt CPM} and {\tt Is_Retail} as
%% {\tt IR}.}. We add the predictions ({\tt SVC, kNN}) and return {\tt Target\_ml} to the
%% dataframe. In order to compare the predictions with the  {\tt Target\_ml} flags, 
%% we  sum the absolute difference for each [e.g. {\tt d\_SVC = abs(Target\_ml - SVC)}] giving the total
%% number of incorrect predictions.

%% Applying the model to the dataset as a whole, we obtain
%% {\color{blue}
%% \footnotesize
%% \begin{verbatim}
%%          e_23456  e_64645   e_43568  CPM  IR   e_11234   e_34454  SVC  kNN  Target_ml  d_SVC  d_kNN
%% 0         986.36     0.40    662.98    1   1   5599.89    283.05    0    0          0      0      0
%% 1       17181.08     3.22  38635.13    1   0   9669.93  20905.39    0    0          0      0      0
%% 2        3036.08     1.11    180.38    1   1  12835.48   4729.43    0    0          0      0      0
%% 3        1764.97     0.38   6059.89    1   1   1452.38  10649.72    0    0          0      0      0
%% 4        1963.98     0.89   5180.09    1   1   6301.98  12770.14    0    0          0      0      0
%% ...          ...      ...       ...  ...  ..       ...       ...  ...  ...        ...    ...    ...
%% 207393   4228.40     1.35   6142.00    2   0   1431.52    858.00    1    1          0      1      1
%% 207394   1877.56     0.25     14.89    2   0   8782.55   2613.07    0    1          0      0      1
%% 207395   5786.43     1.66  36966.07    2   0   8628.77  15898.21    1    0          0      1      0
%% 207396  16485.09     3.15  37475.16    1   0   7522.49  37475.16    0    0          0      0      0
%% 207397   3589.55     1.22  31473.88    1   0   9229.69    578.35    0    0          0      0      0

%% [207398 rows x 12 columns]

%% SVC: 26816 incorrect out of 207398, giving 87% accuracy
%% kNN: 20035 incorrect out of 207398, giving 90% accuracy
%% \end{verbatim}
%% \normalsize
%% }
%% These scores are higher than quoted above, but due to the small number of suspect players we cannot avoid
%% contaminatign these data with the training sample.\footnote{The full results are given in {\tt ML\_predictions.csv}}

%% CAN'T AVOID HAVING TRAINING DATA IN 

%% JUST SHOW THE TARGET ML =1 RESULTS? 

%\subsection{Future Prospects}

Lastly, we note that the sample is small, being trained on just $2\times1752$ players.  Increasing
the sample size is expected to increase the accuracy and, in order to investigate this, using {\tt
  ML\_sampled.py}, we show the mean scores from 100 trials for each of the algorithms, with the top
seven kNN features for randomly selected samples of varying size.\footnote{Plotted with {\tt
  sampled\_plot.py}}

\includegraphics[angle=0,scale=0.65]{media/sampled_plot.eps}

This suggests that the accuracy does not increase significantly for sample sizes above
1000 suspect and 1000 non-suspect players, although the spread in the scores narrows and the
DTC classifier is trending strongly upwards.  In any case, a significant increase in accuracy cannot be
ruled out until much larger datasets are used for the training.

 \end{document}

